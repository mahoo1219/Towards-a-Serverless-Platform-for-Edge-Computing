\section{Introduction}

%The original problem
The advent of real-time and data-intensive applications empowered by mobile and Internet of Things (IoT) devices 
at the network edge 
poses challenges to the centralized data centre model. The network latency from edge devices to cloud data centres is prohibitive for most real-time and interactive applications.
%, preventing computation offloading from resource-constrained edge devices to distant cloud servers. 
Moreover, the transport and analysis of exponentially larger volumes of data by centralized services may result in network bottlenecks and consequently low throughput. 

%Edge/fog computing solution
Targeting the aforementioned problems, fog computing~\cite{Bonomi:2012} --- also known as edge computing, among other similar concepts~\cite{Satyanarayanan:2009,Taleb:2013,ETSI:MEC:2016:03} --- aims to fill the gap between centralized data centres and applications at the network edge with a dense geographical distribution of computing and storage resources.

The intrinsic challenges for its realization brought attention from the research community, whose contributions focused on various aspects such as the management of fog node resources~\cite{N.Wang:2017}; the placement and migration of application components and services onto fog and cloud nodes~\cite{Wang:2015a,Wang:2017,Machen:2018}; and the scheduling of computation offloading from mobile and IoT devices~\cite{Liu:2016, OrsiniBL16}. 
Nonetheless, in many works the concept of an fog node --- or platform, using ETSI's terminology~\cite{ETSI:MEC:2016:03} --- remain abstract; the fulfillment of different application scenarios demands further clarification from the architecture, implementation, and operations perspectives.

Among the capabilities offered by distinct fog deployment configurations, densely distributed nodes converge in their limitation of resources. This characteristic poses the challenge of maximizing the efficiency in which resources are used by fog applications. For instance, the conventional provisioning of \textit{Virtual Machines} may significantly limit the number of concurring applications and users in the system.

%One 

%In this work, we focus on the materialization of an edge platform.



%fine-grained nodes requires an efficient management of resources to render edge-based solutions feasible and scalable.

%For this, the resources must be allocated when actually needed in an automated way. 



%that does not exhibit virtually unlimited resources as cloud data centres~\cite{}. 

%Notwithstanding the benefits of serverless computing, the application scenarios of edge computing have particular needs that requires the optimization of the existing FaaS model and technology. Also, it requires additional services and mechanisms composing the building blocks of a platform able to satisfy edge application needs.

In the recent years, serverless computing has been proposed as an alternative execution model for cloud computing~\cite{Lloyd18serverless}. In a serverless architecture, infrastructure management is fully delegated to third party providers, who takes care of dynamically provisioning and allocating resources --- thus the name \textit{serverless}. The \textit{Function-as-a-Service} model realizes a serverless architecture by allowing application logic, written as stateless functions, to be executed on demand by containerized runtime environments without pre-allocating resources~\cite{Roberts:2018}. 

In this paper, we address the main application scenarios of edge computing with a \textit{Serverless Edge Platform}. The paper contributions are threefold. Firstly, we discuss the benefits of the adoption of a serverless architecture and the \textit{Function-as-a-Service} model in the materialization of a platform. Secondly, distinct application scenarios are presented along with 
%details about its specific requirements and 
the platform services addressing their needs. Finally, we extended \textit{OpenWhisk} --- a state-of-art FaaS platform --- with tools and optimizations composing a \textit{Serverless Edge Platform} prototype. 
%TODO
The proposed services were evaluated in terms of its resource utilization footprint, latency overhead, throughput, and scalability in a resource-constrained fog deployment. Results demonstrated the feasibility of the proposal in satisfying the requirements from distinct application scenarios.

%we discuss the need for different platform services composing the building blocks of a serverless edge platform; finally, we evaluate a platform prototype composed of state-of-art tools satisfying the identified needs.

The remainder of the paper is organized as follows. Section~\ref{sec:background} presents the main characteristics of serverless computing and the \textit{Function-as-a-Service} model.
%and the function-as-a-service model justifying their adoption. 
Throughout Section~\ref{sec:SEP}, the building blocks of the \textit{Serverless Edge Platform} are presented and motivated by means of different application scenarios. Section~\ref{sec:prototype} presents the architecture and tools composing the platform prototype, whilst Section~\ref{sec:evaluation} reports on the evaluation results. Section~\ref{sec:related_works} surveys related works. Finally, Section~\ref{sec:conclusions} draws conclusions and present our future works.

%. A serverless platform for edge computing must address these needs with the optimization of the existing FaaS model and technology along with the additional services and mechanisms targeting different application scenarios and edge computing infrastructures.

%The benefits of adopting a serverless architecture with edge computing to enable low-latency applications has been discussed elsewhere~\cite{}. 

%Notwithstanding this, details on its realization are still missing. 
%target the materialization of such architecture by proposing an edge platform based on the serverless architecture. In addition to mobile computation offloading, the proposed platform also targets the in-transit analysis of data-intensive applications by edge nodes.


%Responsibility over infrastructure and resource provisioning is transferred to providers.